\documentclass[10pt, oneside]{article}
\usepackage{amsmath}
\usepackage{amssymb}
\usepackage[utf8]{inputenc}
\usepackage[english]{babel}
\usepackage{titling}
\usepackage[nottoc, notlof]{tocbibind}
\usepackage[pdftex]{graphicx}
\usepackage[kerning,spacing]{microtype}
\usepackage{verbatim}
\usepackage{tikz}
\usetikzlibrary{arrows}

\usepackage[bookmarksnumbered, unicode, pdftex]{hyperref}

\author{Mikhail Glushenkov, \texttt{<c05mgv@cs.umu.se>}\\
        Bertil Nilsson, \texttt{<id09bnn@cs.umu.se>}}

\title{Assignment 2 -- GCom Middleware:\\Final Report}

\newcommand{\unit}[1]{\ensuremath{\, \mathrm{#1}}}

\begin{document}
\pagestyle{plain}
\pdfbookmark[1]{Front page}{beg}

\begin{titlingpage}
  \begin{minipage}[t]{0.45\textwidth}
  \begin{flushleft}
  \texttt{5DV020 - Distributed Systems, Autumn 12}
  \end{flushleft}
  \end{minipage}
  \begin{minipage}[t]{0.4\textwidth}
  \begin{flushright}
  \texttt{Umeå University}
  \end{flushright}
  \end{minipage}
  \vskip 60pt
  \begin{center}
  \LARGE\thetitle
  \par\end{center}\vskip 0.5em
  \begin{center}
  \large\theauthor
  \par\end{center}
  \begin{center}
  Date: \today
  \par\end{center}
  \vfill
  \begin{center}
    \textbf{Instructors} \linebreak \linebreak
    Francisco Hernandez-Rodriguez\\
    Ewnetu Bayuh Lakew
  \end{center}
\end{titlingpage}

% TOC
%\thispagestyle{empty}
%\pagebreak
%\setcounter{page}{0}
%\pdfbookmark[1]{Table of contents}{tab}
%\tableofcontents
\pagebreak

% % i Sverige har vi normalt inget indrag vid nytt stycke
\setlength{\parindent}{0pt}
% men däremot lite mellanrum
\setlength{\parskip}{10pt}

\setcounter{section}{-1}

\section{Introduction}

The purpose of this assignment was to design and implement GCom, a middleware for
group communication. Middleware is a software layer that provides a high-level
interface to some functionality and frees the user from worrying about how said
functionality is implemented. Group communication is a mode of communication
that allows a set of nodes distributed over network to form groups and broadcast
messages to all members of the groups they belong to.

This is the final report, written at the completion stage.

\section{User's Guide}

The executable files can be found on the department's computer systems, in the
directory \texttt{/c/c05/c05mgv/5dv020}. The source code for the assignment can
be found on GitHub under the address
\texttt{https://github.com/23Skidoo/5dv020}.

To run the name server, use the following command:
\begin{center}\texttt{gcom-nameserver [-p PORT]}\end{center}

The name server expects an RMI registry to be running on local host on the port
\texttt{PORT}.

To run the combined test/debugging application, use the following command:
\begin{center}\texttt{gcom-client -p PORT -c RELIABILITY -o ORDERING -l NODES NAMESERVER COMMAND}\end{center}

Here, \texttt{NAMESERVER} is the address of the name server in the form of a
\texttt{name:host:port} triple, and the \texttt{COMMAND} parameter is one of
\texttt{list}, \texttt{kill NAME} or \texttt{join NAME}. The first command lists
the available groups, the second sends a ``kill'' message to the given group,
and the third joins a given group and starts the graphical interface. The
\texttt{-c} option sets the reliability mode (one of \texttt{\{reliable,
  unreliable\}}), and the \texttt{-o} option sets the ordering mode (one of
\texttt{\{unordered, fifo, causal, total, causaltotal\}}). The \texttt{-l} option
specifies how many nodes need to have joined before operation starts, if left
out or zero means dynamic group structure. The \texttt{-p} option has the same
meaning as for the name server.

\section{Requirements Specification}

This assignment is divided into three levels - the obligatory basic level and
two bonus levels that give additional points. Our solution implements the Level
2 specificaton (basic level + dynamic groups).

The GCom middleware consists of three logical modules: communication, message
ordering and group management. Additionally, we are required to implement a GUI
test application for showcasing the features of the GCom middleware and a GUI
debug application for demonstrating that it works correctly.

The communication module supports two operations: basic non-reliable multicast
and basic reliable multicast. The type of multicast and the message type are set
at module initialisation time, so only a single send operation is accessible at
runtime. This operation takes a set of node identifiers and a message as input,
and blocks until the message has been placed into the incoming message queues of
all of the recipients (it is synchronous in this sense). The communication
module also allows to register callbacks for acting upon a message delivered to
the current node -- this is used to pass control to the next layer of code.

The message ordering module is built on top of the communication module. It
allows to deliver messages according to several different orderings. The
following orderings are supported: non-ordered, FIFO, causal, total and
causal-total (described in more detail in the assignment specification and in
the textbook\cite{Textbook}). Again, the type of ordering is set at module
initialisation time. The same callback mechanism for passing control to the next
layer is used here.

The group management module is the user-facing part of the system implemented on
top of the previous two modules. It allows to create and remove groups, add and
remove group members, handles the monitoring of live nodes, keeps track of
changes in group membership, and allows to list the names of all existing
groups.

The test program is a simple distributed GUI chat program in which each chat
client instance is a node of the distributed system. It is combined

The debug application is implemented as a special client for the aforementioned
chat system that allows to watch the inner workings of the middleware. It has
the following features:
\begin{itemize}
\item Simulated packet loss (send a message to a part of the group).
\item Simulated packet rearrangement (send several messages in random order to
  different nodes).
\item Simulated packet delay (broadcast a message, but delay dispatching to some
  nodes).
\item Creation of a group with a chosen message ordering and multicast type.
\item Display of the internal state of the system (such as message queues).
\item Measurement of system performance.
\end{itemize}

\section{Design}

We decided to implement our system in Scala and use Java RMI through Scala
interop for communication between nodes. The choice of Scala was motivated by
the desire to use a language that is more modern and convenient than Java, but
still allows to run on JVM. Java RMI was chosen because it was the default
option (recommended in the assignment specification) and we felt that other
messaging solutions that we looked at either didn't provide any benefit for our
use case (Finagle) or did all the required work for us (ZeroMQ).

The communication module is just a straightforward implementation of the
textbook algorithms for reliable and unreliable multicast\cite{Textbook}. The
same can be said about the message ordering module: FIFO ordering will be
implemented with logical timestamps, vector clocks will be used for the causal
ordering, total ordering will be implemented with a sequencer node (specified as
one of the inputs to the send function), and causal-total is just a combination
of causal and total.

\begin{figure}[h]
\centering
\includegraphics[width=12cm]{graph1}
\caption{Group management module.}
\label{fig:group}
\end{figure}

The group management module has a slightly more complicated design. Each node is
identified by a (host, port number, name) triple. A group is identified by its
name; names are globally unique, which is enforced by the name server that also
associates an ID of a coordinator node with each group name. The central server
is used a hub through each new nodes can discover and join existing groups.

The coordinator node in each group serves as a message sequencer and gateway
through which new nodes join the group. The process of joining a group is
illustrated in figure~\ref{fig:group}. A node first contacts the central server,
which knows the names of all groups and all coordinator nodes. When a node has
chosen which group it wants to join, it contacts the coordinator for that group
and receives the list of all group members. The group members then synchronise
its view of group membership and the node is considered admitted.

If the coordinator node dies or leaves the group, a new coordinator is elected,
and all group members and the central server are notified. The 2-phase commit
protocol is used to ensure that all nodes have a consistent view of group
membership, group coordinator and the value of the total ordering
counter. Originally, we planned to use Paxos, but that turned out to be too
onerous to implement, and 2PC was fine for our purposes (fail-stop failure
model). Since the focus of this assignment is on multicast and message ordering,
we feel fine with a simplified solution.

In case a group gets partitioned, each partition will elect a new coordinator
and will continue to function independently. The ones that can't contact the
central server will know that they are the ones ``left behind''. They can choose
to either shut down, continue to communicate among themselves or wait for the
central server to become available and try to rejoin the group.

The system is designed to scale fairly well, since at no point there is a need
to communicate with or store the names of all nodes in the system. Each node
communicates only with the members of the groups it belongs to, and the central
server communicates only with the coordinator nodes. The central server is
obviously a single point of failure, but the groups don't depend on it to
function and we believe that a central group catalog is needed in any case so
that new nodes can discover what groups exist in the system.
\pagebreak
\subsection{Modules}
The Group module of GCom is exposed to the Application layer, but only the 
methods defined in the Communicator interface. On its creation the application
layer provides the appropriate GCom settings as well as a callback to execute
when a message arrives. 

Each layer in our system hooks into the lower layer by providing a callback 
to call when a message is ready for the next level. To send a message the 
application layer calls the sendToAll method on the group module which then 
passes the message down in the stack. Each module adds the specific information 
needed as the message passes through. The different layers and modules can 
be seen in figure \ref{fig:modules}

\begin{figure}[h]
\centering
\includegraphics[height=12cm]{graph2}
\caption{Overview of the system}
\label{fig:modules}
\end{figure}


\subsection{Debugger}

The graphical debugging is implemented by using the Publisher/Reactor model in
the Scala Swing wrapper. Each component extends the Publisher trait so that the
graphical user interface can listen to that component. When the components have
something new to push to the interface they publish an event containing that
specific information. 

\section{Limitations}

The nameserver component does not periodically check that group leaders are still
alive. This can lead to dead groups lingering in the system if a 
group crashes and no new leader is chosen by the group to update the status. 

Messages sent out of order after a node has joined the network can be discarded
if the first message received by the node is not the first message sent. The
ordering modules will discard on individual criteria.

\pagebreak

\begin{thebibliography}{9}

\bibitem{Textbook} \emph{Distributed Systems}\\
\newblock George Coulouris, Jean Dollimore, Tim Kindberg and Gordon Blair\\
\newblock Addison-Wesley, 2011\\

\end{thebibliography}


\end{document}
